\documentclass{../../../kin_math}

\header{Elijah Kin}{Homework 1}{AMSC660}
\headrule

\begin{document}

\begin{questions}
  \question (Problem 7 from Section 2.10 in D. Bindel's and J. Goodman's book ``Principles of Scientific Computing'') Starting with the declarations
  \begin{verbatim}
      float x, y, z, w;
const float oneThird = 1/ (float) 3;
const float oneHalf = 1/ (float) 2;
// const means these never are reassigned\end{verbatim}
  we do lots of arithmetic on the variables \texttt{x}, \texttt{y}, \texttt{z}, \texttt{w}. In each case below, determine whether the two arithmetic expressions result in the same floating point number (down to the last bit) as long as no \texttt{NaN} or \texttt{inf} values or denormalized numbers are produced.
  \begin{enumerate}
    \item \begin{verbatim}
( x * y ) + ( z - w )
( z - w ) + ( y * x )\end{verbatim}
    \begin{solution}
      Yes, these expressions result in the same floating point number; as mentioned in Section 2.2.2, floating point addition and multiplication are commutative.
    \end{solution}
    \item \begin{verbatim}
( x + y ) + z
  x + ( y + z )\end{verbatim}
    \begin{solution}
      No, these expressions do not necessarily result in the same floating point number. This is demonstrated by the following C code:
      \begin{lstlisting}[language=C]
        bool is_nonassociative(float x, float y, float z) {
          float lhs = (x + y) + z;
          float rhs = x + (y + z);
          printf("lhs=%.8f\nrhs=%.8f\n", lhs, rhs);
          return lhs != rhs;
        }

        int main() {
          // Addition isn't associative
          assert(is_nonassociative(1.0, 0.0000002, 0.0000003));
          return 0;
        }
      \end{lstlisting}
    \end{solution}
    \item \begin{verbatim}
  x * oneHalf + y * oneHalf
( x + y ) * oneHalf\end{verbatim}
    \begin{solution}
      Yes, these expressions result in the same floating point number; consider the floating point representations of \texttt{x}, \texttt{y}:
      \begin{itemize}
        \item $\texttt{x} = (-1)^{s} \cdot (1.b_1b_2\dots b_d) \cdot 2^e$
        \item $\texttt{y} = (-1)^{s'} \cdot (1.b_1'b_2'\dots b_d') \cdot 2^{e'}$
      \end{itemize}
      Multiplication by \texttt{oneHalf} is done exactly by decrementing the exponent, and hence
      \begin{itemize}
        \item $\texttt{x * oneHalf} = (-1)^{s} \cdot (1.b_1b_2\dots b_d) \cdot 2^{e - 1}$
        \item $\texttt{y * oneHalf} = (-1)^{s'} \cdot (1.b_1'b_2'\dots b_d') \cdot 2^{e' - 1}$
      \end{itemize}
      Without loss of generality, assume $e < e'$, so
      \begin{multline*}
        \texttt{x * oneHalf + y * oneHalf} \\
        = 2^{e - 1}\left((-1)^{s} \cdot (1.b_1b_2\dots b_d) + (-1)^{s'} \cdot (1.b_1'b_2'\dots b_d') \cdot 2^{e' - e}\right).
      \end{multline*}
      On the other hand,
      \begin{equation*}
        \texttt{x + y} = 2^e\left((-1)^{s} \cdot (1.b_1b_2\dots b_d) + (-1)^{s'} \cdot (1.b_1'b_2'\dots b_d') \cdot 2^{e' - e}\right)
      \end{equation*}
      and so multiplying by \texttt{oneHalf} is done exactly by decrementing the exponent to yield the same result as the other expression.
    \end{solution}
    \item \begin{verbatim}
  x * oneThird + y * oneThird
( x + y ) * oneThird\end{verbatim}
    \begin{solution}
      No, these expressions do not necessarily result in the same floating point number. This is demonstrated by the following C code:
      \begin{lstlisting}[language=C]
        bool is_nondistributive(float x, float y, float z) {
          float lhs = x * z + y * z;
          float rhs = (x + y) * z;
          printf("lhs=%.8f\nrhs=%.8f\n", lhs, rhs);
          return lhs != rhs;
        }

        int main() {
          // Multiplication isn't distributive
          const float oneThird = 1 / (float)3;
          assert(is_nondistributive(1.000003, 0.000002, oneThird));
          return 0;
        }
      \end{lstlisting}
    \end{solution}
  \end{enumerate}

  \question The \emph{tent map} of the interval $[0, 1]$ onto itself is defined as
  \begin{equation}
    f(x) = \begin{cases} 2x, & x \in [0, \half) \\ 2 - 2x, & x \in [\half, 1]. \end{cases}
  \end{equation}
  Consider the iteration $x_{n + 1} = f(x_n)$, $n = 0, 1, 2, \dots$.
  \begin{enumerate}
    \item What are the fixed points of this iteration, i.e. the points $x^*$ such that $x^* = f(x^*)$? Show that these fixed points are unstable, i.e., if you start iteration at $x^* + \delta$ for any $\delta$ small enough then the next iterate will be farther away from $x^*$ then $x^* + \delta$.
    \begin{solution}
      Suppose $x^* \in [0, 1]$ is such that $x^* = f(x^*)$. It is either the case that $x^* \in [0, \half)$ or $x^* \in [\half, 1]$. If $x^* \in [0, \half)$, then $x^* = 2x^*$, so $x^* = 0$. Otherwise, $x^* = 2 - 2x^*$, so $x^* = 2 / 3$. Hence, the fixed points of this iteration are $0$ and $2 / 3$.

      We will now show that these fixed points are unstable; consider first the case of $x^* = 0$. Then for $0 < \delta < 1 / 2$,
      \begin{equation*}
        f(x^* + \delta) = f(\delta) = 2\delta
      \end{equation*}
      which is clearly further from $0$ than $\delta$, so $0$ is unstable.

      Now consider $x^* = 2 / 3$. For $-1 / 6 \leq \delta \leq 1 / 3$, then $x^* + \delta \in [\half, 1]$, so
      \begin{equation*}
        f(x^* + \delta) = f(2 / 3 + \delta) = 2 - 2(2 / 3 + \delta) = 2/3 - 2\delta
      \end{equation*}
      which again is clearly further from $2 / 3$ than $2 / 3 + \delta$, so $2 / 3$ is also unstable.
    \end{solution}
    \item Prove that if $x_0$ is rational, then the sequence of iterates generated starting from $x_0$ is periodic.
    \begin{solution}
      Let $x_0 \in [0, 1]$ be rational, in which case there exist $a_0, b \in \mathbb{N}$ such that $x_0 = a_0 / b$. We will prove by induction that every $x_n$ can be written as $a_n / b$ with $a_n \in \mathbb{N}$ such that $0 \leq a_n \leq b$.

      As the base case, we have that $0 \leq x_0 \leq 1$. Rewriting $x_0$ as $a_0 / b$ and multiplying by $b$ yields $0 \leq a_0 \leq b$, and hence the claim holds for $n = 0$.

      As the inductive step, suppose $x_n$ can be written as $a_n / b$ with $0 \leq a_n \leq b$. Then note
      \begin{equation*}
        x_{n + 1} = f(x_n) = \begin{cases} 2x_n, & x_n \in [0, \half) \\ 2 - 2x_n, & x_n \in [\half, 1] \end{cases} = \begin{cases} 2a_n/b, & a_n \in [0, b/2) \\ 2(b - a_n)/b, & a_n \in [b/2, b] \end{cases}
      \end{equation*}
      and hence
      \begin{equation*}
        a_{n + 1} = \begin{cases} 2a_n, & a_n \in [0, b/2) \\ 2(b - a_n), & a_n \in [b/2, b] \end{cases}
      \end{equation*}
      In the first case, $0 \leq a_n < b / 2$ and so multiplying by 2 yields that $0 \leq 2a_n = a_{n + 1} < b$. In the second case, $b / 2 \leq a_n \leq b$ and so multiplying by $-1$ yields
      \begin{equation*}
        -b \leq -a_n \leq -b / 2.
      \end{equation*}
      Then adding $b$ yields
      \begin{equation*}
        0 \leq b - a_n \leq b / 2
      \end{equation*}
      and finally multiplying by $2$,
      \begin{equation*}
        0 \leq 2(b - a_n) = a_{n + 1} \leq b
      \end{equation*}
      so the claim holds in both cases, and hence $x_{n + 1}$ can be written as $a_{n + 1} / b$ with $0 \leq a_{n + 1} \leq b$. By the principle of induction, the claim holds for all $n \in \mathbb{N}$.

      Finally, note that there are only finitely many values $a_n$ such that $0 \leq a_n \leq b$; in particular, $b + 1$ many. Hence, it must be that the sequence of iterates is periodic, as it can take only finitely many values.
    \end{solution}
    \item Show that for any period length $p$, one can find a rational number $x_0$ such that the sequence of iterates generated starting from $x_0$ is periodic of period $p$.
    \begin{solution}
      Let $p \in \mathbb{N}$. We claim that $x_0 \coloneqq \frac{2}{2^p + 1}$ generates a periodic sequence of iterates of period $p$. By part (b) the sequence of iterates must be periodic.

      To see that this sequence has period $p$, note that
      \begin{equation*}
        x_0 = \frac{2}{2^p + 1} < \frac{2^2}{2^p + 1} < \dots < \frac{2^{p - 2}}{2^p + 1} < \frac{2^{p - 1}}{2^p + 1} < \frac{1}{2} < \frac{2^p}{2^p + 1}
      \end{equation*}
      and hence the sequence will double until reaching $x_{p - 1} = \frac{2^p}{2^p + 1}$. However,
      \begin{equation*}
        f(x_{p - 1}) = 2 - 2x_{p - 1} = \frac{2}{2^p + 1} = x_0
      \end{equation*}
      so the sequence of iterates is periodic with terms $\{x_0, x_1, x_2, \dots, x_{p - 2}, x_{p - 1}\}$, and hence is of period $p$.
    \end{solution}
    \item Generate several long enough sequences of iterates on a computer using any suitable language (Matlab, Python, C, etc.) starting from a pseudorandom $x_0$ uniformly distributed on $[0, 1)$ to observe a pattern. I checked in Python and C that 100 iterates are enough. Report what you observe. If possible, experiment with single precision and double precision.
    \begin{solution}
      From the Python script below, we find that after about $53$ iterations, the sequence reaches $1$, and then $f(1) = 0$. Finally, since $f(0) = 0$, the sequence remains $0$ thereafter.

      \begin{lstlisting}[language=Python]
        import random

        # Tent map on the interval [0, 1]
        def f(x):
          return (2 * x) if (x < 0.5) else 2 - (2 * x)

        def iterate(n):
          x = random.random()
          print(x)
          for i in range(n):
            x = f(x)
            print(x)

        iterate(55)
      \end{lstlisting}
    \end{solution}
    \item Explain the observed behavior of the generated sequences of iterates.
    \begin{solution}
      One bit of precision is lost at each iteration by multiplying by $2$ until eventually all precision is lost.
    \end{solution}
  \end{enumerate}
\end{questions}

\end{document}
